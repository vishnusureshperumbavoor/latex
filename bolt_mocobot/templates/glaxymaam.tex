\documentclass[a4paper,11pt]{report}
 
 \usepackage[left=3cm, right=3cm, top=3cm, bottom=3cm]{geometry}
\usepackage{graphicx}
\usepackage{listings}
\usepackage{titlesec}
\usepackage{fancyhdr}
\usepackage{epstopdf}
\usepackage{float}
\usepackage{amsmath}
\usepackage{setspace}
\usepackage{eufrak}
\usepackage{url}

\usepackage{courier}
 \newcommand{\textform}[1]{\fontsize{14}{20}\selectfont{#1}}
\pagestyle{fancy}
\fancyhf{}
\fancyhead[R]{\thepage}
\renewcommand{\chaptermark}[1]{\markboth{#1}{}}
\renewcommand{\headrulewidth}{1pt}
\renewcommand{\footrulewidth}{1pt}

\lhead{\footnotesize{Web And Personal Image Annotation By Mining Label
Correlation With Relaxed Visual Graph Embedding
}}
\rhead{}
\lfoot{\footnotesize{Department of Computer Science \& Engg., FISAT}}
\cfoot{}
\rfoot{\thepage}

\titleformat{\chapter}[display]
{\normalfont\Large\bfseries\centering}{\chaptertitlename\
\thechapter}{20pt}{\Large}


%\title{\textbf{A seminar report on\\TRUSTWORTHINESS MANAGEMENT IN THE SOCIAL INTERNET OF THINGS} 
%}
%\author{\textbf{Deena Jose}}

\begin{document}

\thispagestyle{empty}
  \begin{center}
    \fontsize{14}{25}\selectfont{\textbf{A SEMINAR REPORT}}\\
    \fontsize{14}{25}\selectfont{on}\\
    \fontsize{18}{25}\selectfont{\textbf{WEB AND PERSONAL IMAGE ANNOTATION BY MINING LABEL
    CORRELATION WITH RELAXED VISUAL GRAPH EMBEDDING
    }}\\[1.5cm]
    \fontsize{12}{25}\selectfont{\textbf{Submitted by}}\\[.2cm]
    \fontsize{14}{25}\selectfont \bfseries{GLAXY GEORGE}\\[.6cm]
%\vfill
    \includegraphics[scale=.3]{log1}\\
    \fontsize{12}{10}\selectfont \textbf{FEDERAL INSTITUTE OF SCIENCE AND TECHNOLOGY (FISAT)}\\[0.25cm]
    \small \bfseries{ANGAMALY-683577, ERNAKULAM (DIST)}\\
    \small {\textit{Affiliated to}}\\
    \large \bfseries{MAHATMA GANDHI UNIVERSITY}\\
    \small \bfseries{Kottayam-686560}\\
    \large \bfseries{2013}\\
  \end{center}


\newpage
  \thispagestyle{empty}
    \begin{center}
      \fontsize{14}{20}\selectfont \textbf{FEDERAL INSTITUTE OF SCIENCE AND TECHNOLOGY (FISAT)}\\
      \fontsize{14}{20}\selectfont {Mookkannoor(P.O), Angamaly-683577}
      
    \fontsize{14}{20}\selectfont \textbf{MAHATMA GANDHI UNIVERSITY, Kottayam- 686560
DEPARTMENT OF COMPUTER SCIENCE \& ENGINEERING}\\[1.5cm]

      \includegraphics[scale=.3]{log1}\\[.7cm]
      \textbf{CERTIFICATE}
    \end{center}
    \vspace{.5cm}
    \textform{This is to certify that this report entitled \textbf{Web And Personal Image Annotation By Mining Label
    Correlation With Relaxed Visual Graph Embedding
    } is a bonafide report of the seminar presented during $1^{st}$ semester by \textbf{ Glaxy George}, in partial fulfillment of the requirements for the award of the degree of Master of Technology (M.Tech) in Computer Science \& Engineering during the academic year 2013 - 2015.}\\[0.15cm]

%\begin{flushleft}
  %\fontsize{14}{20}\selectfont {Staff in Charge}
  \hfill
  \fontsize{14}{20}\selectfont {Head of the Department}
%\end{flushleft}

\vspace{0.5cm}
\begin{flushleft}
  \fontsize{12}{20}\selectfont \textbf{Date  :}\\
  \fontsize{12}{20}\selectfont \textbf{Place : Mookkannoor}
\end{flushleft}

\newpage
\fontsize{12pt}{20}\selectfont
\thispagestyle{empty}
  \renewcommand\abstractname{\textform{\textbf{ACKNOWLEDGMENT}}}
    \begin{abstract}
      \vspace{2.5cm}
      
         If the words were considered as symbols of approval and token of acknowledgement, then let
the words pay the heralding role of expressing my gratitude. First and Foremost i praise the God
almighty for the grace he showered on me during my studies as well as my day-to-day life
activities.\\
\newline
    \hspace{2.5cm}       I would like to take this chance to thank the Principal Dr.K.S.M.Panikar, and Chairman of
fisat, Mr.P.V.Mathew for providing me with such an environment, where students can explore
their creative ideas. Equally eligible is the Head of the department of computer science, Dr.
J.C.Prasad for encouraging the students to make these nations true.\\
\newline
   \hspace{1.5cm}        I am extremely grateful to the seminar guide,Dr.Arun Kumar,  Professor in Computer Science and Engineering Department, for his valuable suggestions for
the seminar. I sincerely thank the computer science and engineering faculty for providing us with
invaluable help.\\
  \hspace{1.5cm}         Last but not the least, i thank all my families and friends for giving me the help, strength and
courage for accomplishing the task.

       
    \end{abstract}


\thispagestyle{empty}
  \renewcommand\abstractname{\textform{\textbf{ABSTRACT}}}
    \begin{abstract}
      \vspace{1.0cm}

 \paragraph{ }The number of digital images rapidly increases, and it becomes an important challenge to organize these resources effectively. As a way to facilitate image categorization and retrieval, automatic image annotation has received much research attention. Considering that there are a great number of unlabeled images available, it is beneficial to develop an effective mechanism to leverage unlabeled images for large-scale image annotation. Meanwhile, a single image is usually associated with multiple labels, which are inherently correlated to each other.A new inductive algorithm is proposed for image annotation by integrating label correlation mining and visual similarity mining into a joint framework.A graph model is constructed according to image visual features. A multilabel classifier is  trained by simultaneously uncovering the shared structure common to different labels and the visual graph embedded label prediction matrix for image annotation.The globally
 optimal solution of the proposed framework can be obtained
 by performing generalized eigen-decomposition. The
 proposed framework is applied to both web image annotation and personal
 album labeling using the NUS-WIDE, MSRAMM 2.0, and Kodak
 image data sets, and the AUC evaluation metric. Extensive experiments
 on large-scale image databases collected from the web
 and personal album show that the proposed algorithm is capable
 of utilizing both labeled and unlabeled data for image annotation
 and outperforms other algorithms.


\end{abstract}


%\maketitle
%\tableofcontents
\thispagestyle{empty}
\listoffigures
\thispagestyle{empty}
\listoftables
\thispagestyle{empty}
\thispagestyle{empty}
  \tableofcontents
\thispagestyle{empty}

\chapter{INTRODUCTION}\label{chp:chapter1}
\thispagestyle{fancy}
\pagenumbering{arabic}
  \setcounter{page}{1}
  \renewcommand{\baselinestretch}{1.50}
\paragraph{}With the development of computer networks and storage technologies,web images has increased. There are large amounts of digital images generated, shared, and accessed on different websites, e.g., Flicker. With the popularity of digital cameras, we are able to create personal photos easily. Consequently, the size of personal albums is getting larger. The growing number of web and personal images requires an effective retrieval and browsing mechanism in either a content- or keyword-based manner. Much research effort has been focused on this area during recent years, resulting in remarkable achievements . Among others, automatic image annotation technology, which associates images with labels or tags, has received much research interest. Automatic
image annotation enables conversion of image retrieval into text matching. Indexing and retrieval of text documents are faster and usually more accurate than that of raw multimedia data. Image annotation thus brings several benefits in image retrieval, such as high efficiency and accuracy.
\paragraph{} As a way to facilitate image categorization and retrieval, automatic image annotation has received much research attention. Considering that there are a great number of unlabeled images available, it is beneficial to develop an effective mechanism to leverage unlabeled images for large-scale image annotation. Meanwhile, a single image is usually associated with multiple labels, which are inherently correlated to each other. A straightforward method of image annotation is to decompose the problem into multiple independent single-label problems, but this ignores the underlying correlations among different labels. In this paper, we propose a new inductive algorithm for image annotation by integrating label correlation mining and visual similarity mining into a joint framework.  


\paragraph{}Image annotation is essentially a classification problem. In
the field of multimedia and computer vision, many researchers
have proposed a variety of machine learning and data mining
algorithms for automatic image annotation recently [1],[2].
These works have shown promising achievements in overcoming
the well-known semantic gap by applying machine
learning algorithms to image annotation. Generally speaking,
these approaches can be roughly divided into the following two
groups:

\paragraph{}The approaches in the first group are usually referred to as a
tagging or retrieval-based paradigm. Image tagging approaches
usually annotate images by leveraging web images, which are
associated with user-defined tags. Typically, tagging approaches
can be divided into two phases, i.e., a searching phase and a
mining-for-tags phase. Tagging approaches first search for similar
images from web-scale data sets and then mine the textual
information associated with the retrieved images for image
annotation.
\paragraph{} Generally, there are three major research issues in
image tagging: First, how to design an efficient indexing and
matching algorithm for fast search over large-scale web image
data sets; second, how to define accurate metrics for the retrieval
process; and, third, how to utilize the search results for image
tagging. For example, in [2], an efficient hashing scheme is proposed
for image tagging. The system in [2] first searches for
semantically and visually similar images from the web and then
annotates images by mining the search results. In [3], a multiple-
feature distance metric learning algorithm was proposed
for cartoon image retrieval. Wu et al. proposed a probabilistic
distance metric learning scheme for retrieval-based image annotation
[4]. Because web images with user-generated tags are
comparatively easy to obtain, image tagging has the advantage
that less human labor is required. However, the automatically
acquired images and tags are essentially noisy and incomplete
[5]. Considering that the performance directly depends on the quality of user-generated tags, which are always unseen to the
system, the performance of the tagging system is not stable. It
remains unclear how retrieval-based image annotation systems
will perform when the number of incorrect tags grows.

\paragraph{}The approaches in the second group are usually referred to as
labeling- or learning-based algorithms. Different from tagging,
labeling usually requires some training images, which are labeled
by human supervisors to learn a classifier for image annotation
[5]. In [5], Cao et al. applied canonical correlation analysis
(CCA)  to web image annotation. Based on -norm
regularization, a semisupervised algorithm was proposed in [6]
for image annotation. A relevance model between image and
word was proposed in [6] for automatic image annotation. Compared
with user-generated tags, the labels of training images are
clean and more reliable. However, a limitation of labeling is
that much human labor may be required to annotate large-scale
image repositories.
\paragraph{}Usually, a single image may be associated with multiple labels,
and the image annotation is a typical multilabel classification
problem. A straightforward way to deal with this problem is
to decompose it into several binary classification problems, with
one for each label. However, the limitation is that this type of
approach does not consider correlations among different class
labels. Intuitively, such information is helpful for us to better
understand image content. For example, the keyword “sea” may
often be accompanied with the keyword “beach.” Such information
is quite helpful to better understand the multimedia semantics
.Thus, another method to reduce the required labor
in image labeling is to utilize the label correlation for image annotation.
In the field of machine learning and data mining, some
researchers have also suggested that incorporating the information
of label correlation into multilabel learning is beneficial for
a reliable classification result. These research efforts
have shown that utilizing class correlation information can improve
the performance of multilabel classification in many domains.
\paragraph{}In Fig. 1. In part A of the
figure, many training images are labeled as “beach” and “sea.”
Ideally, the system should learn a pattern that there is a strong
relationship between “beach” and “sea.” Then, for the training
image in part B, which is labeled as “beach” and “sunset,” the
system additionally labels it as “sea.” The unlabeled training
image in part C, which is visually similar to the image in part B, is then labeled as “beach,” “sunset,” and “sea.” In that way,both the label correlation and visual information are considered
for image annotation during training.
\begin{figure}
\begin{center}
\fbox{\includegraphics[scale=1.1]{figure1.png}}
\caption{Illustration of semisupervised multilabel learning for image annotation.}
\label{fig:butterfly}
\end{center}
\end{figure}
\paragraph{}A new framework is proposed for automatic
web and personal image labeling by integrating shared structure
learning (SSL) and graph-based learning into a joint framework.
Compared with other existing algorithms, our algorithm simultaneously
utilizes the information in the unlabeled data and the
label correlation information.


\paragraph{}The rest of this report is organized
as follows:In chapter 2 a brief review of related works is presented. In chapter 3 details of proposed framework, whereas in chapter
4 theoretical discussion of the relationship
between proposed framework and some other related approaches.In chapter 5 and 6 presents
 extensive experiments and  conclusion respectively.

\chapter{RELATED WORK}

\paragraph{}In this section, some related algorithms are discussed. 
\paragraph{}

\section{Shared Structure
Learning (SSL)}
\paragraph{}
Real-world problems usually exhibit dual-heterogeneity, i.e. every task in the problem has features from multiple views, and multiple tasks are related with each other through one or more shared views. To solve these multi-task problems with multiple views, a shared structure learning framework is proposed, which can learn shared predictive structures on common views from multiple related tasks, and use the consistency among different views to improve the performance. This paper suggests a method for multiclass learning with many classes by simultaneously learning shared characteristics common to the classes, and predictors for the classes in terms of these characteristics. This problem is casted  as a convex optimization problem, using trace-norm regularization and study gradient-based optimization both for the linear case and the kernelized setting. 
\paragraph{}
The challenge of accurate classification of an instance into one of a large number of target classes surfaces in many domains, such as object recognition, face identification, textual topic classification, and phoneme recognition. In many of these domains it is natural to assume that even though there are a large number of classes (e.g. different people in a face recognition task), classes are related and build on some underlying common characteristics. For example, many different mammals share characteristics such as a striped texture or an elongated snout, and people’s faces can be identified based on underlying characteristics such as gender, being Caucasian, or having red hair. Recovering the true underlying characteristics of a domain can significantly reduce the effective complexity of the multiclass problem, therefore transferring knowledge between related classes.
\paragraph{}
Simultaneously learning the underlying structure between the classes and the class models is a challenging optimization task. Many of the heuristic approaches explored in the past aim at extracting powerful non-linear hidden characteristics. However, this goal often entails non-convex optimization tasks, prone to local minima problems. In contrast, modeling the shared characteristics, as linear transformations of the input space. Thus, this model will postulate a linear mapping of shared features, followed by a multiclass linear classifier.
\paragraph{}
In [7],  suggests a method for multiclass
learning with many classes by simultaneously
learning shared characteristics common to the
classes, and predictors for the classes in terms of
these characteristics. To address this as a convex optimization
problem, using trace-norm regularization
and study gradient-based optimization both
for the linear case and the kernelized setting.
\paragraph{}
Multi-task learning (MTL) aims to improve
generalization performance by learning multiple
related tasks simultaneously. In [8],  considers the problem of learning
shared structures from multiple related
tasks. An improved formulation
(iASO) is used for multi-task learning based on the
non-convex alternating structure optimization
(ASO) algorithm, in which all tasks are
related by a shared feature representation.
 iASO is converted, a non-convex formulation,
into a relaxed convex one, which is, however,
not scalable to large data sets due to its complex
constraints. An alternating
optimization (cASO) algorithm is proposed which solves
the convex relaxation efficiently, and further
show that cASO converges to a global optimum.
In addition,  a theoretical
condition is proposed, under which cASO can find a globally
optimal solution to iASO.
\paragraph{}
Tagged Web images provide an abundance of labeled training examples for visual concept learning. However, the performance of automatic training data selection is susceptible to highly inaccurate tags and typical images. Consequently, manually curated training data sets are still a preferred choice for many image annotation systems. This paper introduces ARTEMIS—a scheme to enhance automatic selection of  training images using an instance-weighted mixture modeling framework. An optimization algorithm is derived to learn  instance-weights in addition to mixture parameter estimation, essentially adapting to the noise associated with each example. The mechanism of hypothetical local mapping is evoked so that data in diverse mathematical forms or modalities can be cohesively treated as the system maintains tractability in optimization. Finally, training examples are selected from top ranked images of a likelihood-based image ranking.
\paragraph{} 
Experiments indicate that ARTEMIS exhibits higher resilience to noise than several baselines for large training data collection. The performance of ARTEMIS-trained image annotation system is comparable with usage of manually curated data sets. In recent years, easy access to loosely labeled Web images has greatly simplified training data selection. Search engines retrieve potential training examples by comparing concept names with image labels (user-assigned tags or surrounding text keywords). In this context, a concept is illustrated by all images labeled with the concept name and an image with multiple labels exemplifies co-occurring concepts. The retrieved images could be directly used to train annotation systems, except that they are often irrelevant from a machine learning perspective. Even user-assigned tags are highly subjective and about 50% have no relation to visual content. 
\paragraph{}

Tags appear in no particular order of relevance and the most relevant tag occurs in top position in less than 10 percentage of the images . Consequently, several strategies have been proposed to refine retrieved collections. Our approach is based on the observation that the distribution of relevant images has a more regular form compared to noise, thereby resulting in a higher signal to noise ratio at the modes of the distribution as opposed to its boundaries. In that case, the precision of training data selection may be enhanced by tapping the high-likelihood region of the distribution. This in turn evokes a causality dilemma because the distribution parameters cannot be robustly determined without suppressing the effect of outliers and outliers cannot be suppressed without a good reference distribution.
\paragraph{}
 A new instance-weighted mixture-modeling scheme that simultaneously estimates mixture parameters and instance weights. It is named ARTEMIS after Automatic Recognition of Training Examples for Modeling Image Semantics. In this parametric scheme, the reference model for each concept is a mixture model of visual and textual features computed from images tagged with the target concept. Similar to K-Means, the ARTEMIS initialization stage assigns equal weights to all data instances. However, it then deviates by systematically learning unequal weights to curb the contribution of noisy images in iterative reference model learning. Training data is selected by ranking images in the decreasing order of mixture likelihood.


\section{Semisupervised Inductive Learning}
Although transductive classification is comparatively
effective for image annotation, it is not suitable for large-scale
image databases whose size grows dynamically. On the one
hand, manually annotating many training data is expensive
and time consuming. On the other hand, insufficient labeled
training data may induce overfitting. To relieve the tedious work
in supervised learning, some researchers suggest improving the
learning performance by leveraging unlabeled data, e.g., [10]
and [11]. Compared with traditional supervised learning algorithms,
such as linear discriminant analysis (LDA), this type
of algorithm is able to reduce the required number of labeled
data during the training stage. Compared with transductive
learning, the inductive algorithm is able to predict the labels of
unseen data, which are outside the training set. It is therefore
more suitable to apply the algorithm to dynamic image database
annotation. However, in most of existing semisupervised
learning algorithms such as [10] and [11], a linear constraint
is imposed on the image labels, whereas data distribution of
multimedia data is demonstrated to be more of a nonlinear
manifold . It is beneficial to make the classifier more
flexible [12]. Moreover, the correlations among different labels
are not considered either.
\paragraph{}
Although it has been studied for years by the computer vision and machine learning communities, image annotation is still far from practical. In this paper, we propose a novel attempt at model-free image annotation, which is a data-driven approach that annotates images by mining their search results. Some 2.4 million images with their surrounding text are collected from a few photo forums to support this approach. The entire process is formulated in a divide-and-conquer framework where a query keyword is provided along with the uncaptioned image to improve both the effectiveness and efficiency. This is helpful when the collected data set is not dense everywhere. In this sense, our approach contains three steps:
\begin{enumerate}
\item the search process to discover visually and semantically similar search results
\item the mining process to identify salient terms from textual descriptions of the search results 
\item the annotation rejection process to filter out noisy terms yielded by Step 2. 
\end{enumerate}
  
To ensure real-time annotation, two key techniques are leveraged—one is to map the high-dimensional image visual features into hash codes, the other is to implement it as a distributed system, of which the search and mining processes are provided as Web services. As a typical result, the entire process finishes in less than 1 second. Since no training data set is required, our approach enables annotating with unlimited vocabulary and is highly scalable
and robust to outliers.
\paragraph{} Recently, some researchers began to leverage Web-scale data for image understanding. Fundamentally, the aim of image auto-annotation is to find a group of keywords  that maximizes the conditional distributions .
\section{Supervised Learning Of Semantic Classes}
\paragraph{}
A probabilistic formulation for semantic image annotation and retrieval is proposed. Annotation and retrieval are posed as classification problems where each class is defined as the group of database images labeled with a common semantic label. It is shown that, by establishing this one-to-one correspondence between semantic labels and semantic classes, a minimum probability of error annotation and retrieval are feasible with algorithms that are 
\begin{enumerate}


\item conceptually simple
\item computationally efficient
\item do not require prior semantic segmentation of training images.
\end{enumerate}
 In particular, images are represented as bags of localized feature vectors, a mixture density estimated for each image, and the mixtures associated with all images annotated with a common semantic label pooled into a density estimate for the corresponding semantic class. This pooling is justified by a multiple instance learning argument and performed efficiently with a hierarchical extension of expectation-maximization. The benefits of the supervised formulation over the more complex, and currently popular, joint modeling of semantic label and visual feature distributions are illustrated through theoretical arguments and extensive experiments. The supervised formulation is shown to achieve higher accuracy than various previously published methods at a fraction of their computational cost. Finally, the proposed method is shown to be fairly robust to parameter tuning.

\section{Feature  Selection  For  Multimedia Analysis}

While much progress has been made to multi-task classification and subspace learning, multi-task feature selection has long been largely unaddressed. In this paper, we propose a new multi-task feature selection algorithm and apply it to multimedia (e.g., video and image) analysis. Instead of evaluating the importance of each feature individually, our algorithm selects features in a batch mode, by which the feature correlation is considered. While feature selection has received much research attention, less effort has been made on improving the performance of feature selection by leveraging the shared knowledge from multiple related tasks. Our algorithm builds upon the assumption that different related tasks have common structures. Multiple feature selection functions of different tasks are simultaneously learned in a joint framework, which enables our algorithm to utilize the common knowledge of multiple tasks as supplementary information to facilitate decision making. An efficient iterative algorithm is proposed to optimize it, whose convergence is guaranteed. Experiments on different databases have demonstrated the effectiveness of the proposed algorithm.






\chapter{PROPOSED FRAMEWORK
}
\paragraph{}To exploit label correlations for image annotation, it is reasonable
to assume that different image labels are related and
built on some underlying common structures. For example, different
photos taken at the beach share common characteristics,
including sea, sky, and sand. Inspired by the recent work of
SSL , it is assumed that there is a common subspace
shared by multiple image labels. The final label of each image
is predicted by its vector representation in the original feature
space, together with the embedding in the shared subspace. Motivated
by the recent success of semisupervised learning 
[10], [11] we construct a graph model according to image
visual features to exploit the unlabeled data and assume that the
distribution of image labels is consistent with it. In that way,
we integrate SSL and graph-based transductive classification
into a joint framework to learn a reliable multilabel classifier.
Note that our framework is inductive, making it applicable to
large-scale dynamic image databases, which grow dynamically.
\paragraph{}

\section{Formulation of Proposed Framework}
\paragraph{}Define the class indicator matrix of the training images as 
\begin{equation}
Y=Y_l^T,Y_u^T \varepsilon \{0,1\}
\end{equation}

Let $y_{il}$ be the $ l^{th}$ element of $y_i$.Denote $x_i$
as the visual feature of the $i_{th}$ image in the training set.If $x_i$
belongs to the $l^{th}$ class,$y_{il}$=1 ; otherwise,$y_{il}$=0 .The prediction function $f_l$ of the $l_{th}$ label is then defined as
\begin{equation}
f_l(x)= v_l^Tx + p_l\varTheta^Tx = w_l^Tx
\end{equation}
where $v_l\varepsilon\Re^{dx1}$ and $p_l\varepsilon\Re^{rx1}$ are weight vectors,$\Theta\varepsilon\Re^{dxr}$
is a transformation matrix of shared low-dimensional subspace,$w_l=v-l+\Theta_{pl}$, and is the dimension of the shared subspace.

\paragraph{}
A visual graph A is constructed according to
image visual features, whose element $A_ij$ reflects the visual
similarity between the two images $x_i$ and $x_j$.  Practically,A can be defined as follows to reduce the
number of parameters:


$A_{ij} =  1$ , if $x_i$ and $x_j$ are k nearest neighbors  and 0 otherwise.

\section{ Optimization}
\paragraph{}Although the proposed framework
 is nonconvex, the global optimum can be
obtained by performing generalized eigendecomposition.In summary, the training process of the
proposed LMGE algorithm is listed here.
\begin{enumerate}
\item Perform principal component analysis to reduce
the dimension of X, in which all the eigenvectors
corresponding to the nonzero eigenvalues of the
covariance matrix is preserved.
\item Compute C 
\begin{equation}
C=I-\beta(XX^T +(\alpha + \beta)I -\mu XB^{-1}X^T)
%^{-1}
\end{equation}
\item Compute D
\begin{equation}
D=N^{-1}XB^{-1}UYY^TUB^{-1}X^TN^{-1}
%^{-1}
\end{equation}
\item Compute W
\begin{equation}
W=(M-\beta\theta\theta^T)^{-1}XF
%^{-1}
\end{equation}
\end{enumerate} 

\section{Implementation Issues
} 
\subsection{Nonsingularity Issue} In the proposed framework need to
compute the inverse of several matrices.A proof of theorem is given to show that all matrices to be inversed during the training
stage are invertible.
\subsection{Computation Complexity}
 At the training
stage,  compute the Laplacian matrix L , of which
the complexity is $dxn^2$ . To optimize the objective function,
we need to compute the inverse of several matrices and perform
eigen-decomposition. The complexity of these operations
is max$(d^3,n^3) $. Note that is usually much larger than d . Thus,
the complexity of training process is about $n^3$. In
the experiment , the image annotation performance is satisfactory
when the size of training images is 10 000. Once W is
obtained,  perform $cxdxv$ times multiplications
to annotate testing images. Therefore, the image annotation
complexity of this framework is approximately linear with respect
to v, making it applicable to large-scale image databases.
The experiment also demonstrated that proposed algorithm can be applied
to large-scale image annotation.


\chapter{DISCUSSIONS
}
\paragraph{}
This section includes the possible extensions of LMGE proposed in this report and the relationship between proposed framework and some other related works.
\section{Extensions Of LMGE}
\paragraph{}In this framework, the least-squares loss is used. Some other loss function, such as hinge loss and logistic loss, could also be used in the framework. However, the optimization for these loss functions is much more complicated. The least-squares loss function achieves comparable performance to other complicated loss functions, provided that appropriate regularization is added. Therefore, employed the least-squares loss as the loss function for its simplicity.

\paragraph{}
Another possible extension of LMGE is to generalize it to a nonlinear algorithm by utilizing kernel tricks. To this end, first map the image data into a Hilbert space and assume that there is a transformation function that assigns each image datum in one or multiple image label(s).Kernelized principal component analysis is performed as preprocessing. Therefore, framework can be easily extended to a nonlinear one.

\section{Relationship To Dimension Reduction Algorithms }
\paragraph{}
During recent years, many dimension reduction algorithms have been applied to image classification and retrieval. For example, SDA and applied it to image classification and retrieval,CCA to image annotation . In the following, the relationship between our framework and some other dimension reduction algorithms:
\paragraph{}
If the orthogonal constraint is removed in  and the data are centered, turns to the objective function of SDA . It is easy to see that LDA is a special case of SDA. In addition, as proved in, CCA reduces to LDA in multiclass learning problems. Therefore, the optimal in framework coincides with the optimal projections of SDA, CCA, and LDA.


\section{Relationship To SSL Algorithms}  
 In this algorithm of SSL has been proposed for multitask learning, whose objective function is to uncover shared structures. The problem is nonconvex, and thus, an iterative approach, i.e., alternating structure optimization (ASO), is proposed to obtain the local optima . More recently, SSL has been applied to multilabel learning.

\section{Relationship To Transductive Classification  } 
Compared with the transductive classification algorithms proposed in related papers and in this framework is able to annotate the images out of the training set. Moreover, the correlations among different image labels are also exploited in  framework, resulting in more accurate labeling results.

\section{Relationship To Traditional Graph Regularization } 
In many traditional inductive  semisupervised learning algorithms, such as SDA  and Laplacian regularized least squares (LRLS), has been frequently employed as graph regularization to exploit the unlabeled data . If  analyze this regularization term under this framework, that a linear constraint is imposed on the graph embedded label prediction matrix . In the framework, however, such linear constraint is relaxed by simultaneously minimizing the linear constraint  terms .This property, i.e., relaxed linear constraint, makes the framework more flexible  and intrinsically different from most of the existing inductive semisupervised learning algorithms.
 
\chapter{EXPERIMENTAL EVALUATION
}
\paragraph{}This section include extensive experiments to test the
performance of the proposed framework in terms of web image
and personal album labeling.
\section{Data Set Description
}
\paragraph{}In  experiments,  two web image data
sets are used, i.e., NUS-WIDE [13] and MSRA-MM 2.0 [14], and one
personal image collection, i.e., the Kodak Consumer Video
Benchmark Data set [15], to test the labeling performance of
the framework. In all of the three image databases, an image
may be associated with more than one label.
\paragraph{}
The NUS-WIDE image database collected by Lab for
Media Search in the National University of Singapore consists
of 269 000 real-world web images crawled from Flickr
[13]. Downloaded all the 269 000 images from
http://lms.comp.nus.edu.sg/research /NUS-WIDE.htm. Among
the 269 000 images, there are 59 653 unannotated images and
209 347 images, which are annotated with ground-truth labels
from 81 concepts.   The unannotated images have removed
and used all of the remaining 209 347 images, along with the
ground-truth labels in the experiments.
\paragraph{}
The MSRA-MM 2.0 image database is collected by Microsoft
Research Asia [14]. In MSRA-MM 2.0, there are
around 1 million web images acquired by Live Image Search
using different predefined queries. The queries are manually
classified into eight categories, i.e., “Animal,” “Cartoon,”
“Event,” “Object,” “Scene,” “PeopleRelated,” “NamedPerson,”
and “Misc” [14]. Among the 1 million images, there are 42 266
images that are manually annotated with ground-truth labels. 3
Each image from the MSRA-MM 2.0 database is labeled as
positive or negative with respect to each concept. There are 100
concepts in total. Detailed information of this database can be
found in [14]. All of the 42 266 annotated images are used
from this data set in the experiment.
\paragraph{}The Kodak Consumer Video Benchmark Data set is collected
by the Eastman Kodak Company from about 100 consumers
over a period of one year. There are 5166 images (keyframes)
extracted from 1358 consumer video clips. Of these images,
3590 from this database are annotated by students from Columbia
University, who are asked to assign binary labels (presence
or absence) for each concept. There are 22 visual concepts
in total. In the experiment,  have used all of the 3590 annotated
images.
\section{Compared Schemes}
\paragraph{}
In the experiment,  the proposed LMGE framework is compared
with baseline and a number of related state-of-the-art algorithms.



\paragraph{}
Employed rigid regression (RR) as the baseline algorithm
in the experiment.The proposed framework is compared with two dimension
reduction algorithms, i.e., CCA , which has been applied
to image annotation in recent work [16], and SDA [17],
which has been applied to image classification and retrieval in
[17]. For CCA and SDA, the two algorithms are first applied to
reducing the dimension of the input visual feature vector, and
then, RR is performed as a classifier. In addition, we compare
our algorithm with the shared structure multitask learning algorithm,
i.e., ASO, .Also compared the framework
with two recently proposed multilabel classification algorithms,
including dimensionality reduction with multilabel classification
 and SSL . To compare
the framework with existing graph-based inductive classification
algorithms,  additionally compared  LMGE with
the well-known semisupervised learning algorithm LRLS  and the Multilabel Learning by Solving a Sylvester
Equation  (denoted as SYLVE in this section).
Because the experiment setting is inductive and transductive algorithms,
such as LGC , are not appropriate for dynamic
image databases, so did not compare proposed algorithm with them.
\section{Experiment Setup and Evaluation Metrics}

In the experiment,  the three types of
visual features are downloaded, including 144-dimension normalized color
correlogram, 128-dimension normalized wavelet texture, and
73-dimension normalized edge direction histogram provided
by the National University of Singapore [13] to represent the
images in the NUS-WIDE data set. Also downloaded the
three types of visual features, including 144-dimension normalized
color correlogram, 128-dimension normalized wavelet
texture, and 75-dimension normalized edge direction histogram
provided by Microsoft Research Asia [14] to represent the
images in MSRA-MM 2.0 data set. Similarly as in [13], 
have used the 144-dimension normalized color correlogram,
128-dimension normalized wavelet texture, and 73-dimension
normalized edge direction histogram to represent the images in
the Kodak data set. The three visual features are  concatenated in
the experiment to represent the images from the three data sets.
The visual feature of each image is centered by subtracting the
mean of the visual features of all the training data.
\paragraph{}
In the experiment, randomly sample n images as training
data, of which m images are labeled. The remaining images are
used as testing images.For   NUS-WIDE
and MSRA-MM 2.0 image databases $n=10000$ are used. Because the Kodak database
is much smaller,  set $n=2000$ for this database. Denote c
as the number of total labels/concepts.Also set m as $5\times c$ ,$10\times c$ 
, and$15\times c$   respectively, and report all of the results. Following
the convention of image annotation, during the sampling
process, each label is guaranteed to appear in at least one image.
The experiments are independently repeated ten times to generate
different training and testing images, and we report the average
results. Statistical significance test is also performed, with
a significance level of 0.05.
\paragraph{}
As   proposed both linear and kernel versions of
the proposed LMGE algorithm, in this experiment,   only
the image annotation results from the linear method for the following
two reasons: First, all the related algorithms to be compared
are linear ones. Second, it is usually a nontrivial task to
select proper kernels for different data sets.
\section{Performance Evaluation}

Fig.5.1 shows several image annotation results of the three
databases, where the top four keywords are used to annotate
the images. It can see from this figure that the proposed algorithm
generally works well for the web image and personal
image databases. Also observes from Fig. 5.1 that there are
some wrong annotation results. For example, the keyword “ski”
is wrongly annotated to two images in the third row. One possible
reason could be that the visual features are not capable
enough to represent the image semantics. An interesting observation
is that, for both cases, “sports” is tagged to the images
as well. Thus, another possible reason might be that the model
overfits the pattern that there is a strong relationship between
“sports” and “ski.” Therefore, how to deal with overfitting could
be a potential research direction in multilabel learning. Next, 
gives the numerical evaluation of the proposed algorithm.
\paragraph{}
As before,c denotes the number of labels/concepts in this
section. Tables I –III show the mean MicroAUC and mean
MacroAUC of ten times independent experiments with standard
deviation of different algorithms when $5\times c$ ,$10\times c$ , and $15\times c$
images are labeled, respectively. A significance test
(t-test) is performed as well. Results that are significantly better
than others are indicated in boldface.
\paragraph{}
First,  Tables $ I - III $  shows that proposed framework  outperforms all of the other algorithms in
terms of mean MicroAUC and MacroAUC. According to the
t-test, the algorithm significantly outperforms all of the other
algorithms in all of the 18 cases. This observation indicates that
the proposed LMGE can effectively learn from both the label
correlations and visual similarities of images.
It can also be found in the three tables that, as the number
of labeled images increases, the performance of all the algorithms
gradually improves. For example, when 405 images
are labeled, the mean MicroAUC of proposed framework for the NUS-Wide image data set is 0.8835. When the number of labeled
images increases to 810 , the mean MicroAUC
of proposed  framework for the NUS-Wide image data set is 0.8976.
If  the number of labeled images increases, it
can be seen that the MicroAUC reaches 0.9035 when 1215
images are labeled. Note that there are 199 347 testing images
in total. Compared with nearly 200 000 testing images, the
number of labeled images is quite small, but the performance
is good. The experiment indicates that it is possible to annotate
large-scale real-world images by only labeling a comparatively
small amount of training images.
\begin{figure} [ht]
 \centering
\includegraphics[height=105mm]{figure2.png}
\caption{Image annotation examples. The wrongly annotated keywords are indicated by red color.}
\end{figure}
\begin{table}[H]
\centering
\tiny\caption{PERFORMANCE COMPARISON (MEAN MICROAUC $\pm$ STANDARD DEVIATION AND MEAN MACROAUC $\pm$ STANDARD DEVIATION) WHEN $5\times c$ IMAGES ARE
LABELED. }
\includegraphics[scale=.7]{table1.png}
\end{table}

\begin{table}[H]
\centering
\tiny\caption{PERFORMANCE COMPARISON (MEAN MICROAUC $\pm$ STANDARD DEVIATION AND MEAN MACROAUC $\pm$ STANDARD DEVIATION) WHEN $10\times c$ IMAGES ARE
LABELED. }
\includegraphics[scale=.7]{table2.png}
\end{table}
\begin{table}[H]
\centering
\tiny\caption{PERFORMANCE COMPARISON (MEAN MICROAUC $\pm$ STANDARD DEVIATION AND MEAN MACROAUC $\pm$ STANDARD DEVIATION) WHEN $15\times c$ IMAGES ARE
LABELED. }
\includegraphics[scale=.7]{table2.png}
\end{table}
\section{Parameter Sensitivity}
There are several parameters to be tuned in the framework.
In  experiment, it is observed that the performance is not sensitive
to the dimension of the shared subspace. When $\mu$ is smaller than 0.01, the performance of LMGE is good for the three image
databases, yet it might be lager for some other data sets. In addition,
also observed that the performance is less sensitive to $\alpha$ and $\beta$
when they are in the range of $[10^{-4},10^2]$.Next, we conduct
experiments to test the performance sensitivity w.r.t. parameter $\kappa$.
\paragraph{}
Fig. 5.2 shows the performance variance with respect to parameter $\kappa$
, which specifies the number of $\kappa$ nearest neighbors to
compute the Laplacian matrix. The number of nearest neighbors
should not be too large. Thus, $\kappa$ is set as {10, 15, 20, 25}
in this experiment.Fig. 5.2 shows that the performance
of LMGE proposed  varies slightly, provided that
is not large. 
\begin{figure} [ht]
 \centering
\includegraphics[height=50mm]{figure3.png}
\caption{Precision–recall curve of image annotation results using different k`s to compute the Laplacian matrix, when $15\times c$ images are labeled. (a) NUS-WIDE.
(b) NSRA-MM 2.0. (c) Kodak.}
\end{figure}
\section{Annotation Time}

Lastly,  tested the efficiency of the proposed framework
for large-scale image annotation. All experiments are run
on a server with 2.67-GHz central processing unit (CPU). 5
The algorithm is implemented by Matlab. Note that, once the
classifier, i.e.,W in (12), is trained,  can use it to compute
the label prediction vector of each testing image according to
(8) for image annotation. As  discussed in Section III, the
time complexity of this process is approximately linear with
respect to the total number of testing images. Table IV shows
the elapsed time (in seconds) to compute the label prediction
vectors of all the testing images after is obtained. The proposed algorithm is very efficient to compute label prediction
vectors for large-scale image data sets. More specifically, the
framework only takes 0.3483 s to compute label prediction
vectors for nearly 200 000 images from NUS-WIDE image
data sets. Therefore, proposed framework can
be applied to annotating large-scale dynamic image data sets.
\begin{figure} [ht]
 \centering
\includegraphics[height=50mm]{table4.png}

\end{figure}





\chapter{CONCLUSIONS
}
\paragraph{}A new framework for web and
personal image annotation is proposed to simultaneously
mine label correlations and visual similarities by integrating
SSL and relaxed visual graph embedding into a joint framework.
Different from previous image annotation algorithms, which
usually learn the classifiers by minimizing the regularized empirical
error, proposed system minimize the prediction error
with respect to a graph embedded label prediction matrix. Compared
with traditional graph-based inductive classification algorithms,
the linear constraint on the label prediction matrix is relaxed,
making it more flexible. Moreover, the label correlation
has been learned by uncovering the shared structure of different
labels.It is shown that, although the problem is nonconvex,
a global optimal solution can be obtained by performing generalized
eigen-decomposition.the connection between our algorithm and
other related works. It is also proved that the algorithm generalizes
several well-known dimension reduction algorithms and
classification algorithms. According to the discussion, proposed framework has provided a new perspective
to analyze traditional graph regularization, which has been
frequently employed in previous semisupervised learning algorithms.
Extensive experiments are conducted in two realworld
web image data sets and one personal image data set. In
these experiments, proposed framework has several
advantages, such as efficiency and accuracy.

\chapter*{BIBILOGRAPHY}
\addcontentsline{toc}{chapter}{BIBLIOGRAPHY}
[1] L. Cao, J. Luo, H. S. Kautz, and T. S. Huang, “Image annotation within
the context of personal photo collections using hierarchical event and
scene models,” IEEE Trans. Multimedia, vol. 11, no. 2, pp. 208–219,
Feb. 2009.\\[0.5cm]
[2] X.-J. Wang, L. Zhang, X. Li, and W.-Y. Ma, “Annotating images by
mining image search results,” IEEE Trans. Pattern Anal. Mach. Intell.,
vol. 30, no. 11, pp. 1919–1932, Nov. 2008.\\[0.5cm]
[3] Y. Yang, Y. Zhuang, D. Xu, Y. Pan, D. Tao, and S. J. Maybank, “Retrieval
based interactive cartoon synthesis via unsupervised bi-distance
metric learning,” in Proc. ACM Multimedia, 2009, pp. 311–320. \\[0.5cm]
[4] L.Wu, S. Hoi, R. Jin, J. Zhu, and N. Yu, “Distance metric learning from
uncertain side information with application to automated photo tagging,”
ACM Trans. Intell. Syst. Technol., vol. 2, no. 2, pp. 13:1–13:28,
Feb. 2011.\\[0.5cm]
[5] L. Cao, J. Yu, J. Luo, and T. S. Huang, “Enhancing semantic and geographic
annotation of web images via logistic canonical correlation
regression,” in Proc. ACM Multimedia, 2009, pp. 125–134.\\[0.5cm]
[6] Z. Ma, Y. Yang, F. Nie, J. Uijlings, and N. Sebe, “Exploiting the entire
feature space with sparsity for automatic image annotation,” in Proc.
ACM Multimedia, 2011.\\[0.5cm]
[7] Y. Amit, M. Fink, N. Srebro, and S. Ullman, “Uncovering shared structures
in multiclass classification,” in Proc. ICML, 2007, pp. 17–24.\\[0.5cm]
[8] J. Chen, L. Tang, J. Liu, and J. Ye, “A convex formulation for learning
shared structures from multiple tasks,” in Proc. ICML, 2009, pp.
137–144.\\[0.5cm]
[9] Y. Yang, F. Nie, D. Xu, J. Luo, Y. Zhuang, and Y. Pan, “A multimedia
retrieval framework based on semi-supervised ranking and relevance
feedback,” IEEE Trans. Pattern Anal. Mach. Intell., 2011, to be published.\\[0.5cm]
[10] D. Cai, X. He, and J. Han, “Semi-supervised discriminant analysis,” in
Proc. ICCV, 2007, pp. 1–7.\\[0.5cm]
[11] M. Belkin, P. Niyogi, and V. Sindhwani, “Manifold regularization: A
geometric framework for learning from labeled and unlabeled examples,”
J. Mach. Learn. Res., vol. 7, pp. 2399–2434, Dec. 2006.\\[0.5cm]
[12]F. Nie, D. Xu, I. W.-H. Tsang, and C. Zhang, “Flexible manifold
embedding: A framework for semi-supervised and unsupervised
dimension reduction,” IEEE Trans. Image Process., vol. 19, no. 7, pp.
1921–1932, Jul. 2010.\\[0.5cm]
[13]T.-S. Chua, J. Tang, R. Hong, H. Li, Z. Luo, and Y.-T. Zheng, “NUSWIDE:
A real-world web image database from national university of
singapore,” in Proc. CIVR, 2009, pp. 1–9.\\[0.5cm]
[14]H. Li, M. Wang, and X.-S. Hua, “MSRA-MM 2.0: A large-scale web
multimedia dataset,” in Proc. ICDMW, 2009, pp. 164–169.\\[0.5cm]
[15]A. Loui, J. Luo, S.-F. Chang, D. Ellis, W. Jiang, L. Kennedy, K. Lee,
and A. Yanagawa, “Kodak’s consumer video benchmark data set: Concept
definition and annotation,” in Proc. MIR, 2007, pp. 245–254.\\[0.5cm]
[16]L. Cao, J. Yu, J. Luo, and T. S. Huang, “Enhancing semantic and geographic
annotation of web images via logistic canonical correlation
regression,” in Proc. ACM Multimedia, 2009, pp. 125–134.\\[0.5cm]
[17]D. Cai, X. He, and J. Han, “Semi-supervised discriminant analysis,” in
Proc. ICCV, 2007, pp. 1–7.
\newpage
\thispagestyle{empty}
\vfill
\fontsize{14}{15}\selectfont.
\vfill
\hfill
\fontsize{12}{15}\selectfont Prepared using \LaTeX \hspace{1pt} 2$\varepsilon$





\end{document}
